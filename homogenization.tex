\chapter{Homogenization}

When modeling geometric and material nonlinearities in heterogeneous materials, analyzing realistic microstructures is critical for achieving high-fidelity predictive capabilities.
Heterogeneous materials may contain a variety of heterogeneous phases (voids, particles, fibers, crystals, etc.) of varying composition, size, and shape arranged in complex spatial distributions (Matouš et al., 2017).
Therefore, it is crucial to develop a representative computational domain of the microstructure to perform accurate and efficient multiscale simulations, i.e., a domain that accurately predicts the effective material behavior (physically meaningful) and preserves the geometrical and morphological complexity of the microstructure (statistically representative).
Hill (1963) was the first to propose the so-called representative volume element (RVE), defined as a volume of material sufficiently large such that it contains enough morphological and topological information about the microstructure heterogeneities to be representative in an average sense.
Given its influence in the solution of the microscale equilibrium problem, the choice of the appropriate boundary conditions to be imposed on the RVE has been a topic of a lengthy discussion, e.g. (Ostoja-Starzewski, 1998; Pecullan et al., 1999; Jiang et al., 2001; Mesarovic and Padbidri, 2005; Shen and Brinson, 2006; Drago and Pindera, 2007; Nilenius et al., 2014).
In particular, the periodic boundary condition is often applied due to its bounded nature (Suquet, 1985; Hollister and Kikuchi, 1992; Nemat-Nasser and Hori, 1995; Terada et al., 2000; Miehe, 2002) and because the resulting homogenized properties converge faster to their real values as the RVE size increases (Terada et al., 2000; Kanit et al., 2003).
The basic assumption behind this boundary condition is that the microstructure can be reproduced by a pattern of identical RVEs.
Therefore, the ability to computationally generate periodic microstructures is especially relevant.
